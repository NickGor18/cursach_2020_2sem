\documentclass[12pt]{article}
\usepackage[utf8]{inputenc}
\usepackage[russian]{babel}
\usepackage{akktex}
\usepackage[left=20mm, top=15mm, right=15mm, bottom=15mm, nohead, footskip=10mm]{geometry} % настройки полей документа

\begin{document}
\begin{center}
\Large{\textbf{Министерство образования Республики Беларусь}}\\
\Large{\textbf{Белорусский Государственный Университет}}\\
\hfill \break
\hfill \break
\large{Факультет прикладной математики и информатики}\
\large{Кафедра методов оптимального управления}\\
\hfill \break
\hfill \break
\hfill \break
\hfill \break
\LARGE{\textbf{Курсовая работа}}\\
\large{\textbf{Методы МРС для отслеживания магистралей в динамических задачах экономики
}}\\
\hfill \break
\hfill \break
\hfill \break
\hfill \break
\hfill \break
\hfill \break
\hfill \break
\end{center}
\begin{flushleft}
	\normalsize{Автор работы: студент (-ка)}\\
	\normalsize{3 курса специализации}\\
	\normalsize{Экономическая кибернетика \underline{\hspace{5cm}}  Н.С.Горошко}\\
	\hfill \break
	\normalsize{Руководитель:}\\
	\normalsize{кандидат физико-математических наук,}\\
	\normalsize{доцент \hspace{4cm} \underline{\hspace{5cm}}   Н.М.Дмитрук}\\
\end{flushleft}
\begin{center}
	\hfill \break
	\hfill \break
	\hfill \break
	\hfill \break
	\hfill \break
	\hfill \break
	\hfill \break
	\hfill \break
	\hfill \break
	\hfill \break
	
	\normalsize{Минск 2020}
\end{center}
\newpage
\Large{\textbf{Оглавление}}
\begin{enumerate}
	\item Глава 1. Обзор литературы.
	\begin{enumerate}
		\item [1.1] Теория управления по прогнозирующей модели
		\item [1.2] Задачи оптимального управления
		\item [1.3] Численные и программные методы решения ЗОУ
	\end{enumerate}
\item Глава 2. Решение задачи.
\begin{enumerate}
	\item [2.1] Математическая модель экономического роста
	\item [2.2] Задача оптимального управления
	\item [2.3] Построение магистралей
	\item [2.4] Стабилизация в окрестности магистрали
\end{enumerate}
\item Глава 3. Результаты численных экспериментов.
\end{enumerate}
\newpage
\section{Обзор литературы}
\subsection{Теория управления по прогнозирующей модели}
\newpage
\subsection{Задачи оптимального управления}
\normalsize{Задачи оптимального управления относятся к теории экстремальных задач, то есть задач определения максимальных и минимальных значений. Задачи эти, как и собственно сама теория оптимального управления, возникла в начале ХХ-го века в связи с практическими задачами, появившимися из-за развития новой техники в различных областях. Данные экстремальные задачи не укладывались в рамки классического вариационного счисления.
	В данной главе мы рассмотрим их, используя различные примеры. В целом решение подобных задач можно разбить на два этапа:
	\begin{enumerate}
		\item Постановка задачи
		\item Решение с использованием условий оптимальности
	\end{enumerate}
Данные пункты содержат в себе сразу несколько подпунктов, так что сейчас мы перейдем от общего к частному.
}
\paragraph{Постановка задачи}
	\hfill \break
\normalsize{Изначально у нас есть некоторое, условие, однако его недостаточно для решения задачи. Для начала проведем математическую постановку задачи. 
	Она в себя будет включать следующие факторы: математическую модель объекта управления, цель управления, ограничения на траекторию воздействия, управляющее воздействие и его длительность и т.д. Рассмотрим данные факторы подробнее.
}\\
\paragraph{Модели объекта}
	\hfill \break
\normalsize{Построение модели зависит от типа рассматриваемой задачи и того, что необходимо в итоге получить. Могут быть использованы различные дифференциальные уравнения: обыкновенные дифференциальные уравнения, уравнения с последействием, стохастические уравнения, уравнения в частных производных и т.д.
	Для примера будем использовать обыкновенное дифференциальное уравнение: }\\
\begin{flushright}
	\normalsize{ $\dot x(t) = f(t,x(t),u),\dot x(t)=dx\dt, t_0 \le t \le T $ \hspace{4cm} (1)}
\end{flushright}
\normalsize{$u \in R^m-$управление, $x \in R^n$-фазовый вектор системы, $f \in R^n$-заданная функция, а $R^n$ – евклидово пространство размерность n. Придавая нашему управлению различные значения мы получаем различные состояния объекта, из которых мы и выбираем оптимальное. }
\paragraph{Критерий качества}
\normalsize{Управление системой (1) осуществляется для достижения некоторых целей, которые формально записываются в терминах минимизации по u функционалов J, определяемых управлением u и траекторией х, где
	\begin{flushright}
		$J= \int_{t_0}^{T}(F(t,x(t),u)dt)+ \varphi (T,x(T)) \rightarrow min$\hspace{4cm}(2)
	\end{flushright} 
F и $\varphi$ – заданные скалярные функции. Задача (2) в общем виде называется задачей Больца. При $F$ = 0 её называют задачей Майера, а при $\varphi$ = 0 – Лагранджа.}
\paragraph{Ограничения на траекторию и ограничения на управление}
\normalsize{Иногда траектория не может принадлежать какой-либо части пространства $R^n$. В таких случаях указывают, что $x(t) \in G(t)$,при том,что $G(t)$-заданная область в $R^n$. В зависимости от типа ограничений выделяют различные классы задач управления, такие как задачи с фиксированными концами, свободным левым либо правым концом. Так же существуют задачи с подвижными концами. Иногда же ограничения имеют интегральный характер и выглядят следующим образом:
	\begin{center}
		$J = \int_{t_0}^{T} F(t,x(t),u)dt \le 0$
	\end{center}
}
\normalsize{Если в задачах (1),(2) начальное и конечное положение задано, моменты начала и конца движения свободны, функция $\varphi = 0$, а $F = 1$, то получаем задачу о переводе системы (1) из начального положения в конечное за минимально возможное время.
Далее мы рассмотри ограничения на управление, а после перейдем к примеру.
Ограничения могут быть двух типов
	\begin{itemize}
		\item Информационные
		\item Ограниченность ресурсов управления
	\end{itemize}
Информационные ограничения на управление зависят от того, какая именно информация о системе (1) доступна при выработке управляющего воздействия. Если вектор $x(t)$ недоступен измерению, то оптимальное управление ищется в классе функций $u(t)$, зависящих только от $t$. В этом случае оптимальное управление называется программным. Если же вектор $x(t)$ известен точно, то оптимальное управление называется синтезом оптимального управления и ищется в классе функционалов $u(t,x_{t_0}^{t}).$ Здесь $x_{t_0}^{t}$ – вся траектория движения на отрезке $t_0 \le s \le t$.
Ограничения, обусловленные ограниченностью ресурсов управления имеют вид $u(t) \in U(t)$,где $U(t)$  заданное множество из $R^m$.
}
\paragraph{Условия оптимальности}
\paragraph{Принцип максимума}
\normalsize{Для начала сформулируем условия оптимальности в общем случае.\\
	\textbf{Теорема}: Пусть $u^0 (t),x^0 (t),t\in T$, - оптимальные управление и траектория задачи
	\begin{center}
		$J(u)=\varphi (x(t^* ))+\int_{0}^{t^*}f_0 (x(t),u(t))dt \rightarrow min$
	\end{center}
\begin{flushright}
	$\dot{x}=f(x,u), x(0)=x_0$ \hspace{6cm} (3)
\end{flushright}
\begin{center}
	$x(t^*) \in X^*=\{x \in R^n: h_i \le 0,i=1,...,m_1,h_i(x)=0,i=m_1+1,...,m\},
	u(t) \in U,t\in T=[0,t^* ]$, где $h_i(x))$-непрерывно дифференцируемые функции,
	$x\in R^n  i=1,...,m, m<n.$
\end{center}
Тогда найдутся такие числа $\lambda_i^0,i=1,...,m)$, что вдоль указанных управления $u^0 (t),t\in T$,траектории $x^0 (t),t\in T$,и решения $\psi^0 (t),t\in T$,сопряженной системы $x(t)^*\in X^*$ выполняются условия: 
}
\begin{center}
	\begin{enumerate}
		\item Условие нетривиальности: $\sum_{i=0}^{m} (\lambda_i^0)^2 \not= 0;$
		\item Условия неотрицательности: $\lambda_i^0 \ge 0, i=0,...,m_1;$
		\item Условие максимума: $H(x^{0}(t),\psi^0(t), u^0(t)) = max H(x^0(t),\psi^0(t), u(t)), t \in \[t,t^*\[, $ где максимум мы берем по $u$
		\item Условие трансверсальности: 
		\begin{center}
			$\psi^0 (t^*)=-\lambda_0^0   \partial\psi(x^0 (t^* ))\partial x-\sum_{i=0}^{m}\lambda_i^0   (\partial h_i (x^0 (t^* )))\partial x$
		\end{center}
		\item Условия дополняющей нежесткости: $\lambda_i^0 h_i (x^0(t^*))=0,i=1,...,m_1.$
	\end{enumerate}
\end{center}
Чтобы продолжить решение нам необходимо понять, что же такое условие максимума. Сформулируем теорему.\\
\textbf{Теорема}: Оптимальное управление управление  $u^0 (t), t\in T $в задаче   $J_p (u)=\varphi(x(t^* ))+\int_{0}^{t^*}f_0(x(t),u(t))dt+\sum_{(i=1)}^{m} \rho_i h_i^2 x(t^* ) \rightarrow min, \dot{x} =f(x,u),x(0)= x_0,  u(t)\in U,t\in T,$
где $\rho_i>0 $– штраф за «единицу» нарушения$ h_i^2 x(t^* )=1$ ограничения $h_i (x)=0$ вместе с соответствующей траекторией $x^0 (t),t\in T$, удовлетворяют условию максимума:$ H(x^0 (t),\psi^0 (t),u^0 (t))=\overset{u\in U}{max}H(x^0 (t),\psi^0 (t),u(t)),t\in \[0,t^* \[,$  где $\psi^0 (t),t\in T$ – решение сопряженной системы $\dot{\psi} =-(\partial H(x(t),\psi,u(t)))/\partial x$ с начальным условием$ \psi^0 (t^* )= -\partial \varphi(x^0 (t^* ))/\partial x -\sum_{(i=1)}^{m}\lambda_i^0   (\partial h_i (x^0 (t^* )))/\partial x$, в котором $\lambda_i^0=2\rho_i h_i x^0 (t^* ),i=1,...,m$.
Данные теоремы используются для решения задач оптимального управления, однако не всегда их использование является эффективным.
\paragraph{Метод динамического программирования}
Применяя метод динамического программирования, мы изучаем все поле оптимальных траекторий. Для того, чтобы сравнение было наглядным, мы воспользуемся ранее заданной задачей (3). Зафиксируем некоторый произвольный момент времени $t \in \[t_0,T\]$. Рассмотрим вспомогательную задачу управления на отрезке $[t ,T]$. Через $V(t,x)$ обозначим минимальное значение критерия качества во вспомогательной задаче при начальном условии $x(t) = x$, где $x$ – произвольный вектор из $R^n$. Мы можем предположить, что функция $V(t,x)$ удовлетворяет соотношениям:
	\begin{center}
		$\partial V(t,x)/\partial t+\overset{u \in U}{min}f^{'} (t,x,u)  \partial V(t,x)/\partial x=0,$
	\end{center}
	\begin{flushright}
	$t_0\le t\le T,x\in R^n,$ \hspace{7cm}                                (4)
	\end{flushright}
	\begin{center}
		$V(T,x)=F(x).$
	\end{center}
	Решив данную задачу и определив $V(t,x)$ мы можем найти управление $u(t,x)$ из соотношения:
	\begin{flushright}
		$\overset{u \in U}{min}f^{'}(t,x,u)  \partial V(t,x)/\partial x=f^{'}(t,x,u(t,x))  \partial V(t,x)/\partial x$ \hspace{4cm}       (5)
	\end{flushright}
	Возможность находить конкретно управление есть характерная черта метода динамического программирования. Она становится особенно важной в условиях отсутствия полной информации. 
	При решении конкретных задач с помощью метода динамического программирования мы решаем нелинейное уравнение в частных производных (4), а так же дополнительно исследуем оптимальное управление, получаемое из уравнения (5).
	\newpage
\subsection{Численные и программные методы решения ЗОУ}
\paragraph{Численные методы}
Здесь будем говорить о четырех численных методах решения задач оптимального управления.
\begin{itemize}
	\item Метод проекции градиента
	\item Метод сопряженного градиента
	\item Метод Ньютона
	\item Метод штрафных функций
\end{itemize}
Рассматривать каждый метод мы не будем, а сфокусируемся исключительно на первом
\subparagraph{Метод проекции градиента}
Опишем алгоритм решения для задачи $f(x) \rightarrow min, x \in X \subset R^n$, где $f(x)$ - непрерывно дифференцируема, а множество $X$ выпукло, замкнуто и ограничено.
Пусть задано начальное приближение $x^0 \in X$ и методом проекции градиента вычислено $x^k \in X$. Следующее приближение вычисляем по формуле
\begin{flushright}
	$x^{k+1}=P_X(x^k-\alpha_k\bigtriangledown f(x^k)), \alpha_k > 0,k=0,1,... \hspace{4cm}(1)$
\end{flushright} 
Сам же алгоритм метода проекции градиента основывается на следующих трех теоремах.

\end{document}